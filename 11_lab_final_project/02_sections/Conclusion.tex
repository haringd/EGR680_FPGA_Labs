\section{Conclusion}\label{sec: Conclusion}
The final project introduces the use of the groove peripherals and the vestal approaches to interface with those. First, the on board leds and RGB led is used to flash and build a color synthesizer that is controlled over a user friendly GUI. Second, python is used to interface the Groove LED bar by which two sliders allows to change the brightness and the level on what value shall be shown. Third, the MicroBlaze is used to use low level C language to build a custom driver for the Groove LED Bar and the Groove Buzzer. The C code is executed in an softcore that interfaces directly with the PMODA connector. With the use of high level python language a simple music synthesizer was build with an user friendly GUI. The C functions of the MicroBlaze can be invoked directly in python,

%The lab demonstrates the use of the ipython as simple and fast scripting language that allows access to vast number of packages that allows an decreased development time. The Jupiter Notebook provides a way to program and develop interactive GUIs that allows partially run code in separate cells.
%
